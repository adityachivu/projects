% Chapter 1

\chapter{Implementation} % Main chapter title

\label{implement} % For referencing the chapter elsewhere, use \ref{Chapter1} 

\lhead{Chapter 4. \emph{Implementation}} % This is for the header on each page - perhaps a shortened title

%----------------------------------------------------------------------------------------

\section{Abstraction of Subroutines}

\section{Code}
Following is the code for seminal models highly illustrative of the architectures presented in earlier chapters.

\subsection*{AlexNet}
\begin{lstlisting}
################################################################################
#Michael Guerzhoy and Davi Frossard, 2016
#AlexNet implementation in TensorFlow, with weights
#Details: 
#http://www.cs.toronto.edu/~guerzhoy/tf_alexnet/
#
#
################################################################################

from numpy import *
import os
from pylab import *
import numpy as np
import matplotlib.pyplot as plt
import matplotlib.cbook as cbook
import time
from scipy.misc import imread
from scipy.misc import imresize
import matplotlib.image as mpimg
from scipy.ndimage import filters
import urllib
from numpy import random


import tensorflow as tf

from caffe_classes import class_names

train_x = zeros((1, 227,227,3)).astype(float32)
train_y = zeros((1, 1000))
xdim = train_x.shape[1:]
ydim = train_y.shape[1]

################################################################################
#Read Image

im1 = (imread("poodle.png")[:,:,:3]).astype(float32)
im1 = im1 - mean(im1)

im2 = (imread("laska.png")[:,:,:3]).astype(float32)
im2 = im2 - mean(im2)

################################################################################

net_data = load("bvlc_alexnet.npy").item()

def conv(input, kernel, biases, k_h, k_w, c_o, s_h, s_w,  padding="VALID", group=1):
    '''From https://github.com/ethereon/caffe-tensorflow
    '''
    c_i = input.get_shape()[-1]
    assert c_i%group==0
    assert c_o%group==0
    convolve = lambda i, k: tf.nn.conv2d(i, k, [1, s_h, s_w, 1], padding=padding)
    
    
    if group==1:
        conv = convolve(input, kernel)
    else:
        input_groups = tf.split(3, group, input)
        kernel_groups = tf.split(3, group, kernel)
        output_groups = [convolve(i, k) for i,k in zip(input_groups, kernel_groups)]
        conv = tf.concat(3, output_groups)
    return  tf.reshape(tf.nn.bias_add(conv, biases), [-1]+conv.get_shape().as_list()[1:])

x = tf.placeholder(tf.float32, (None,) + xdim)


#conv1
k_h = 11; k_w = 11; c_o = 96; s_h = 4; s_w = 4
conv1W = tf.Variable(net_data["conv1"][0])
conv1b = tf.Variable(net_data["conv1"][1])
conv1_in = conv(x, conv1W, conv1b, k_h, k_w, c_o, s_h, s_w, padding="SAME", group=1)
conv1 = tf.nn.relu(conv1_in)

#lrn1
#lrn(2, 2e-05, 0.75, name='norm1')
radius = 2; alpha = 2e-05; beta = 0.75; bias = 1.0
lrn1 = tf.nn.local_response_normalization(conv1,
                                                  depth_radius=radius,
                                                  alpha=alpha,
                                                  beta=beta,
                                                  bias=bias)

#maxpool1
k_h = 3; k_w = 3; s_h = 2; s_w = 2; padding = 'VALID'
maxpool1 = tf.nn.max_pool(lrn1, ksize=[1, k_h, k_w, 1], strides=[1, s_h, s_w, 1], padding=padding)


#conv2
k_h = 5; k_w = 5; c_o = 256; s_h = 1; s_w = 1; group = 2
conv2W = tf.Variable(net_data["conv2"][0])
conv2b = tf.Variable(net_data["conv2"][1])
conv2_in = conv(maxpool1, conv2W, conv2b, k_h, k_w, c_o, s_h, s_w, padding="SAME", group=group)
conv2 = tf.nn.relu(conv2_in)


#lrn2
radius = 2; alpha = 2e-05; beta = 0.75; bias = 1.0
lrn2 = tf.nn.local_response_normalization(conv2,
                                                  depth_radius=radius,
                                                  alpha=alpha,
                                                  beta=beta,
                                                  bias=bias)

#maxpool2                                               
k_h = 3; k_w = 3; s_h = 2; s_w = 2; padding = 'VALID'
maxpool2 = tf.nn.max_pool(lrn2, ksize=[1, k_h, k_w, 1], strides=[1, s_h, s_w, 1], padding=padding)

#conv3
k_h = 3; k_w = 3; c_o = 384; s_h = 1; s_w = 1; group = 1
conv3W = tf.Variable(net_data["conv3"][0])
conv3b = tf.Variable(net_data["conv3"][1])
conv3_in = conv(maxpool2, conv3W, conv3b, k_h, k_w, c_o, s_h, s_w, padding="SAME", group=group)
conv3 = tf.nn.relu(conv3_in)

#conv4
k_h = 3; k_w = 3; c_o = 384; s_h = 1; s_w = 1; group = 2
conv4W = tf.Variable(net_data["conv4"][0])
conv4b = tf.Variable(net_data["conv4"][1])
conv4_in = conv(conv3, conv4W, conv4b, k_h, k_w, c_o, s_h, s_w, padding="SAME", group=group)
conv4 = tf.nn.relu(conv4_in)


#conv5
k_h = 3; k_w = 3; c_o = 256; s_h = 1; s_w = 1; group = 2
conv5W = tf.Variable(net_data["conv5"][0])
conv5b = tf.Variable(net_data["conv5"][1])
conv5_in = conv(conv4, conv5W, conv5b, k_h, k_w, c_o, s_h, s_w, padding="SAME", group=group)
conv5 = tf.nn.relu(conv5_in)

#maxpool5
k_h = 3; k_w = 3; s_h = 2; s_w = 2; padding = 'VALID'
maxpool5 = tf.nn.max_pool(conv5, ksize=[1, k_h, k_w, 1], strides=[1, s_h, s_w, 1], padding=padding)

#fc6
fc6W = tf.Variable(net_data["fc6"][0])
fc6b = tf.Variable(net_data["fc6"][1])
fc6 = tf.nn.relu_layer(tf.reshape(maxpool5, [-1, int(prod(maxpool5.get_shape()[1:]))]), fc6W, fc6b)

#fc7
fc7W = tf.Variable(net_data["fc7"][0])
fc7b = tf.Variable(net_data["fc7"][1])
fc7 = tf.nn.relu_layer(fc6, fc7W, fc7b)

#fc8
fc8W = tf.Variable(net_data["fc8"][0])
fc8b = tf.Variable(net_data["fc8"][1])
fc8 = tf.nn.xw_plus_b(fc7, fc8W, fc8b)


#prob
prob = tf.nn.softmax(fc8)

init = tf.initialize_all_variables()
sess = tf.Session()
sess.run(init)

t = time.time()
output = sess.run(prob, feed_dict = {x:[im1,im2]})
################################################################################

#Output:
for input_im_ind in range(output.shape[0]):
    inds = argsort(output)[input_im_ind,:]
    print "Image", input_im_ind
    for i in range(5):
        print class_names[inds[-1-i]], output[input_im_ind, inds[-1-i]]

print time.time()-t
\end{lstlisting}
\subsection*{Char-rnn for sentence creation}
\begin{lstlisting}
# code from https://github.com/sherjilozair/
from __future__ import print_function
import numpy as np
import tensorflow as tf

import argparse
import time
import os
from six.moves import cPickle

from utils import TextLoader
from model import Model

def main():
    parser = argparse.ArgumentParser()
    parser.add_argument('--data_dir', type=str, default='data/tinyshakespeare',
                       help='data directory containing input.txt')
    parser.add_argument('--save_dir', type=str, default='save',
                       help='directory to store checkpointed models')
    parser.add_argument('--rnn_size', type=int, default=128,
                       help='size of RNN hidden state')
    parser.add_argument('--num_layers', type=int, default=2,
                       help='number of layers in the RNN')
    parser.add_argument('--model', type=str, default='lstm',
                       help='rnn, gru, or lstm')
    parser.add_argument('--batch_size', type=int, default=50,
                       help='minibatch size')
    parser.add_argument('--seq_length', type=int, default=50,
                       help='RNN sequence length')
    parser.add_argument('--num_epochs', type=int, default=50,
                       help='number of epochs')
    parser.add_argument('--save_every', type=int, default=1000,
                       help='save frequency')
    parser.add_argument('--grad_clip', type=float, default=5.,
                       help='clip gradients at this value')
    parser.add_argument('--learning_rate', type=float, default=0.002,
                       help='learning rate')
    parser.add_argument('--decay_rate', type=float, default=0.97,
                       help='decay rate for rmsprop')                       
    parser.add_argument('--init_from', type=str, default=None,
                       help="""continue training from saved model at this path. Path must contain files saved by previous training process: 
                            'config.pkl'        : configuration;
                            'chars_vocab.pkl'   : vocabulary definitions;
                            'checkpoint'        : paths to model file(s) (created by tf).
                                                  Note: this file contains absolute paths, be careful when moving files around;
                            'model.ckpt-*'      : file(s) with model definition (created by tf)
                        """)
    args = parser.parse_args()
    train(args)

def train(args):
    data_loader = TextLoader(args.data_dir, args.batch_size, args.seq_length)
    args.vocab_size = data_loader.vocab_size
    
    # check compatibility if training is continued from previously saved model
    if args.init_from is not None:
        # check if all necessary files exist 
        assert os.path.isdir(args.init_from)," %s must be a a path" % args.init_from
        assert os.path.isfile(os.path.join(args.init_from,"config.pkl")),"config.pkl file does not exist in path %s"%args.init_from
        assert os.path.isfile(os.path.join(args.init_from,"chars_vocab.pkl")),"chars_vocab.pkl.pkl file does not exist in path %s" % args.init_from
        ckpt = tf.train.get_checkpoint_state(args.init_from)
        assert ckpt,"No checkpoint found"
        assert ckpt.model_checkpoint_path,"No model path found in checkpoint"

        # open old config and check if models are compatible
        with open(os.path.join(args.init_from, 'config.pkl')) as f:
            saved_model_args = cPickle.load(f)
        need_be_same=["model","rnn_size","num_layers","seq_length"]
        for checkme in need_be_same:
            assert vars(saved_model_args)[checkme]==vars(args)[checkme],"Command line argument and saved model disagree on '%s' "%checkme
        
        # open saved vocab/dict and check if vocabs/dicts are compatible
        with open(os.path.join(args.init_from, 'chars_vocab.pkl')) as f:
            saved_chars, saved_vocab = cPickle.load(f)
        assert saved_chars==data_loader.chars, "Data and loaded model disagree on character set!"
        assert saved_vocab==data_loader.vocab, "Data and loaded model disagree on dictionary mappings!"
        
    with open(os.path.join(args.save_dir, 'config.pkl'), 'wb') as f:
        cPickle.dump(args, f)
    with open(os.path.join(args.save_dir, 'chars_vocab.pkl'), 'wb') as f:
        cPickle.dump((data_loader.chars, data_loader.vocab), f)
        
    model = Model(args)

    with tf.Session() as sess:
        tf.initialize_all_variables().run()
        saver = tf.train.Saver(tf.all_variables())
        # restore model
        if args.init_from is not None:
            saver.restore(sess, ckpt.model_checkpoint_path)
        for e in range(args.num_epochs):
            sess.run(tf.assign(model.lr, args.learning_rate * (args.decay_rate ** e)))
            data_loader.reset_batch_pointer()
            state = sess.run(model.initial_state)
            for b in range(data_loader.num_batches):
                start = time.time()
                x, y = data_loader.next_batch()
                feed = {model.input_data: x, model.targets: y}
                for i, (c, h) in enumerate(model.initial_state):
                    feed[c] = state[i].c
                    feed[h] = state[i].h
                train_loss, state, _ = sess.run([model.cost, model.final_state, model.train_op], feed)
                end = time.time()
                print("{}/{} (epoch {}), train_loss = {:.3f}, time/batch = {:.3f}" \
                    .format(e * data_loader.num_batches + b,
                            args.num_epochs * data_loader.num_batches,
                            e, train_loss, end - start))
                if (e * data_loader.num_batches + b) % args.save_every == 0\
                    or (e==args.num_epochs-1 and b == data_loader.num_batches-1): # save for the last result
                    checkpoint_path = os.path.join(args.save_dir, 'model.ckpt')
                    saver.save(sess, checkpoint_path, global_step = e * data_loader.num_batches + b)
                    print("model saved to {}".format(checkpoint_path))

if __name__ == '__main__':
    main()
\end{lstlisting}



